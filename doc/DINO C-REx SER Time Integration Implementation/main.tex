\documentclass[]{DINOReportMemo}
\usepackage{DINO_C-REx}
\usepackage{colortbl}


\newcommand{\ModuleName}{Image Generation} %edit this!
\newcommand{\subject}{Systems Engineering Report: Magnitude to Stellar Flux Conversion} %edit this!
\newcommand{\status}{Initial Version}
\newcommand{\preparer}{Matt Muszynski} %edit this!
\newcommand{\summary}{Description, analysis, and recommendation for converting the stellar magnitudes presented in the Tycho-2 catalog to blackbody curves for each star modeled.} %edit this
\usepackage{float}
\usepackage{rotating}
\usepackage{pdflscape}

\begin{document}

\makeCover
%
% enter the revision documentation here
% to add more lines, copy the table entry and the \hline, and paste after the current entry.
%
\pagestyle{empty}
{\renewcommand{\arraystretch}{2}
\noindent
\begin{longtable}{|p{0.5in}|p{4.5in}|p{1.14in}|}
\hline
{\bfseries Rev}: & {\bfseries Change Description} & {\bfseries By} \\
\hline
1.0 & Initial Release & Matt Muszynski \\ %edit this
\hline

\end{longtable}
}

\newpage
\setcounter{page}{1}
\pagestyle{fancy}

\tableofcontents
~\\ \hrule ~\\

\newpage
\section{Scope}
The scope of this SER is to describe the current solution to time integration of images in the camera module of DINO C-REx. Time integration was developed in order to meet needs levied by requirements 4.8.4, 4.8.4.1, and 4.8.4.2. This document also describes the object model of the camera model as it pertains to the time integration problem.
\section{Object Model}
To accomplish time integration appropriately, the DINO C-REx camera team has developed four classes to best organize data throughout the integration process. Of the four classes, the only one with interfaces to items external to the camera model is the Camera class. 

\subsection{The Camera Class}

\subsubsection{External Inputs}
\begin{enumerate}
    \item Spacecraft Position
    \item Spacecraft Attitude
    \item Integration Time Step
    \item Take Image Flag
\end{enumerate}
\subsubsection{External Outputs}
\begin{enumerate}
    \item Detector Array
\end{enumerate}
\subsubsection{User Inputs}
The Camera class has been designed to allow the DINO C-REx user to parameterize the phsical characterization of a spacecraft camera or star tracker. The primary control for expressing this parameterization within the system it the set of user inputs to the Camera class. They should be set in the Python script used to wrap Basilisk.

\begin{table}[]
\centering
\caption{My caption}
\label{my-label}
\begin{tabular}{|l|l|l|l|}
\hline
Name              & Description                                                                                            & Data Type                                 & Constraints                                                                                                                                                                                                                                                                                                   \\ \hline
focal\_length     & The physical focal length                                                                              & Scalar                                    & Can be any unit, but must match alpha and beta.                                                                                                                                                                                                                                                               \\ \hline
a                 & Physical height of detector                                                                            & Scalar                                    & Can be any unit, but must match focal\_length and beta.                                                                                                                                                                                                                                                       \\ \hline
b                 & Physical width of detector                                                                             & Scalar                                    & Can be any unit, but must match focal\_length and alpha.                                                                                                                                                                                                                                                      \\ \hline
alpha\_resolution & Number of pixels in the same direction as the physical height (a)                                      & Scalar                                    & None.                                                                                                                                                                                                                                                                                                         \\ \hline
beta\_resolution  & Number of pixels in the same direction as the physical width (b)                                       & Scalar                                    & None.                                                                                                                                                                                                                                                                                                         \\ \hline
body2cameraDCM    & Direction cosine matrix describing the orientation of the camera relative to the spacecraft body axes. & 3x3 Numpy Array                           & Must be valid direction cosine matrix. The first axis of the camera frame must point along the camera boresight. The second axis will then point left in the FOV, and the third up.                                                                                                                           \\ \hline
max\_mag          & The dimmest magnitude visible to the spacecraft camera.                                                & Scalar                                    & None. DINO C-REx reduced Tycho-2 Catalog has no stars dimmer than magnitude 10, so any max\_mag\textgreater10 will do nothing.                                                                                                                                                                                \\ \hline
tc                & Parameterization of the transmission curve of the lens used for this camera.                           & Python Dict containing 2 1xn Numpy Arrays & \begin{tabular}[c]{@{}l@{}}One entry in the dictionary must be called 'lambda', and the other must be called 'throughput'. Both arrays must be the same length.\\ 'lambda' arrays do not need to match between tc and qe. The interpolate\_lambda\_dependent() method will take care of that.\end{tabular}    \\ \hline
qe                & Parameterization of the quantum efficienct curve of the lens used for this camera                      & Python Dict containing 2 1xn Numpy Arrays & \begin{tabular}[c]{@{}l@{}}One entry in the dictionary must be called 'lambda', and the other must be called 'throughput'. Both arrays must be the same length.\\ \\ 'lambda' arrays do not need to match between tc and qe. The interpolate\_lambda\_dependent() method will take care of that.\end{tabular} \\ \hline
lambda\_bin\_size & The final bin size for interpolated tc, qe, and blackbody curves.                                      & Scalar                                    & None. Making this smaller will give higher fidelity integration, but will increase camera initialization time.                                                                                                                                                                                                \\ \hline
\end{tabular}
\end{table}
\subsubsection{Internal Inputs}
None.
\subsubsection{Internal Outputs}
None.
\subsubsection{Attributes}
None.
\subsubsection{Methods}
None.

\subsection{The Image Class}
\subsubsection{External Inputs}
None.
\subsubsection{External Outputs}
None.
\subsubsection{Internal Inputs}
None.
\subsubsection{Internal Outputs}
None.
\subsubsection{Attributes}
None.
\subsubsection{Methods}
None.

\subsection{The Frame Class}
\subsubsection{External Inputs}
None.
\subsubsection{External Outputs}
None.
\subsubsection{Internal Inputs}
None.
\subsubsection{Internal Outputs}
None.
\subsubsection{Attributes}
None.
\subsubsection{Methods}
None.

\subsection{The Scene Class}
\subsubsection{External Inputs}
None.
\subsubsection{External Outputs}
None.
\subsubsection{Internal Inputs}
None.
\subsubsection{Internal Outputs}
None.
\subsubsection{Attributes}
None.
\subsubsection{Methods}
None.
\section{Section 3}


\end{document}
